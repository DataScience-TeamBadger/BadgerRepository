\documentclass[12pt]{article}
\usepackage[T1]{fontenc}
\usepackage[utf8]{inputenc}
\usepackage[english]{babel}
\usepackage[margin=0.5in]{geometry}
\usepackage{multicol}
\setlength{\columnsep}{1cm}

\title{\textbf{Predictive Modeling for Public Transit Development:}\\Providing Mass Transportation to Minority Groups During Development Planning}

\author{
	von der Lippe, Michael \\ \		\texttt{msvonderlippe@smcm.edu}
	\and
	Sahdeo, Deonarine\\ \		\texttt{dsahdeo@smcm.edu}
	\and
	DeMay, John \\ \		\texttt{jtdemey@smcm.edu}
	\and
	Kerzner, Alex\\ \		\texttt{ajkerzner@smcm.edu}
	\and
	Badger, Stuart\\ \		\texttt{sbadger@smcm.edu}
	\and
	Stephen, Proctor\\ \		\texttt{ineedyouremail@smcm.edu}
	}

\date{St.Mary's College of Maryland\\4/28/2017}


\begin{document}
	\maketitle

\begin{multicols}{2}

\section{ABSTRACT}
Public Transportation is an important major social issue as the movement of people from one place to another allows them access to societal necessities such as proper education, adequate and fulfilling work, as well as many other public services. The allocation of spending on public transportation has shifted towards a focus on suburban movement, with large expansive highways and long railways, which provide little to no advantages for the minority groups within the metropolitan center. This paper justify the problem we will be attempting to solve, provide a predictive model that uses data from three major and distinct cities, in order to estimate the best way to allocate budget during the planning of future urban development projects. The end result of this study will be an application that provides predictions based on our training set, as well as a template method for cleaning up future transportation data in order to strengthen the training set information for more accurate and representative models. 

\section{JUSTIFICATION}
Throughout the twenty-first century, we have seen public transportation play an increasingly important role in developing societies with economic vitality. Mass transit can be broken down into two parts: the movement of people and the movement of goods. On the one hand, multinational corporations like Google and Amazon have studied the transportation of goods, looking for ways to improve that system. On the other hand, not much research has been done into the movement of people despite recent advancements in data science and data analytics. The Civil Rights Project at Harvard performed research to show that “where people live can greatly affect what types of transportation options are available to them to travel to work and to carry out their daily activities.” This is supported by the claim that “since 1960, people of color have increasingly populated metropolitan areas. Only 52 of the 100 largest cities have a majority white population, according to the 2000 census data” (Moving to Equity, 7). 
	With this information, we decided that society’s need for efficient public transportation may not be fulfilled by our current infrastructure. “People’s income levels generally correspond with their ability to own a car and the type of transportation they use” (Moving to Equity, 9). This provoked the thought that efficient public transportation would be able to benefit minorities and may assist in providing equal advantages when attempting to gain access to the working world. As shown in this figure, public transportation use is much higher among minority groups. This means that as we improve public transportation, the impact of those improvements is much more valuable to the community who uses it. 
As a team, we began sifting through some of the many available data sets on public transportation. We started analyzing public transportation infrastructure in aspiration of spotting inefficiencies in transportation. We found that there were three monolithic providers of public transport service: taxis, buses, and underground rail systems such as metro and subway. We decided that a taxi service, although public, is for personal consumption, not mass transit. Therefore, the two services we are most interested in are the public bus system and the metro system (subway). We began to ask questions about what could be done to improve the implementation of these two services. The final question that we reached is: depending on the population density, which is more efficient, underground rail or public bus routes?
We predict that there will be a trend in our data that will show as population density increases, larger systems of transportation infrastructure, such as metro, will be more cost-effective than smaller systems such as bus routes. Conversely, as population density decreases, we predict that bus routes will be more cost-effective than larger systems of transportation. The goal of our research is to first distinguish the effectiveness between the two systems based on the average annual spending, usage, transportation coverage, and cost for citizens. The second goal of our research is to create a predictive model to properly allocate spending on public transportation infrastructure to minimize annual spending (and in turn the money spent by citizens for public transport), while simultaneously maximizing transportation coverage and hopefully improving public transportation usage. By finding a way to improve public transportation infrastructure, we can provide a better transit system for minority groups which would allow them to achieve higher economic vitality and a more fulfilling social life.
\section{CURRENT APPROACHES}

\subsection{Problems with current approaches}

\section{DEFINITIONS AND VARIABLES}
This section pertains to all relevant factors identified during research of the current transportation infrastructures, and the guidelines we set for proper measurement.
\subsection{Definitions}
Due to a lack of preconceived methodology for analyzing and understanding transportation information, the following definitions are those pertinent to the understanding of our projects goals and measurement specifications.\\
1.) Transportation Efficiency - The total coverage of a transportation system, the amount of  in comparison to the number of 
\subsection{Variables}

\section{Data Collection and Collection Methods}

\subsection{Data Collection Errors}

	\end{multicols}
\end{document} 